%************************************************
\section{Beamer}
\begin{frame}\frametitle{Beamer简介}
Beamer 是一个用于制作演示文稿的LaTeX 文档类,由Till
Tantau 编写。相对于其它同类工具,Beamer 有如下这些优点:
\begin{itemize}
    \item<2-> \textcolor{red}{标准统一}。标准的 \LaTeX{} 指令在 Beamer 文稿中可直接使用;
    \item<3-> \textcolor{red}{主题丰富}。提供了许多主题 (theme),可以很容易改善简档的外观;
    \item<4-> \textcolor{red}{注重内容}。致力于更好的表现演讲内容,而不是仅仅为了让页面好看;
    \item<5-> \textcolor{red}{方便定制}。页面布局、色彩、字体都可以实现全局调控;
\end{itemize}
\end{frame}

\begin{frame}[fragile]\frametitle{hello\_beamer.tex}
\begin{block}{}
    \begin{verbatim}
\documentclass{beamer}
\begin{document}
\begin{frame}
Hello Beamer!
\end{frame}%
\end{document}
    \end{verbatim}
\end{block}
从这个例子可以看出,Beamer 中每张幻灯片的内容都是放置在一个frame 环境里面的。
\end{frame}

\begin{frame}[fragile]\frametitle{hello\_beamer\_CN.tex}
\begin{block}{}
    \begin{verbatim}
\documentclass{beamer}
\usepacakge[UTF8]{ctex}
\begin{document}
\begin{frame}
你好Beamer!
\end{frame }
\end{document}
    \end{verbatim}
\end{block}
对于中文文档,建议用UTF8 编码,然后用xelatex 程序编译。
另外,可以在载入ctex 宏包时加上noindent 选项以取消段落
的缩进。
\end{frame}

\begin{frame}[fragile]\frametitle{Beamer基本语法}
\begin{block}{}
    \begin{verbatim}
\begin{frame}{幻灯片标题}{我是一个副标题}
Hello Beamer!
\end{frame }
    \end{verbatim}
\end{block}
或者
\begin{block}{}
    \begin{verbatim}
\begin{frame}
\frametitle{幻灯片标题}
\framesubtitle{我是一个副标题}
Hello Beamer!
\end{frame }
    \end{verbatim}
\end{block}
\end{frame}

\begin{frame}[fragile]\frametitle{Beamer文档结构}
在Beamer文档中,最常用的分节命令是\textbackslash section:
\begin{block}{}
    \begin{verbatim}
\section{Section Name}
    \end{verbatim}
\end{block}
类似于标题页面,我们可以在幻灯片中用\textbackslash tableofcontents 命令生成目录页。
\begin{block}{}
    \begin{verbatim}
\begin{frame}
\tableofcontents[hideallsubsections]
\end{frame }
    \end{verbatim}
\end{block}
其中hideallsubsections 选项表示不显示小节标题。
\end{frame}

\begin{frame}[fragile]\frametitle{简单的列表环境}
\begin{block}{}
    \begin{verbatim}
\begin{itemize}
\item<1-> 我是第一项
\item<2-> 我是第二项
\item<3-> 我是第三项
\end{itemize}
    \end{verbatim}
\end{block}
\begin{block}{}
\begin{itemize}
\item<1-> 我是第一项
\item<2-> 我是第二项
\item<3-> 我是第三项
\end{itemize}
\end{block}
\end{frame}

\begin{frame}[fragile]\frametitle{简单的区块环境}
\begin{block}{}
    \begin{verbatim}
\begin{block}{区块环境}
区块环境为了突出显示某些内容。
\end{block}
    \end{verbatim}
\end{block}
\begin{block}{区块环境}
区块环境为了突出显示某些内容。
\end{block}
\begin{alertblock}{警示区块环境}
警示区块环境为了警示突出某些内容。
\end{alertblock}
\begin{exampleblock}{例子区块环境}
例子区块环境为了突出例子内容。
\end{exampleblock}
\end{frame}

\begin{frame}\frametitle{beamer themes}
Beamer 的整体主题包含了结构、颜色、字体各方面的设置。
\begin{block}{}
\textbackslash usebeamertheme\{主题名\}
\end{block}
\begin{description}
    \item[无导航栏] 无导航栏default、boxes、Bergen、Pittsburgh 和 Rochester。
    \item[带顶栏] Antibes、Darmstadt、Frankfurt、JuanLesPins、Montpellier 和Singapore。
    \item[带底栏] Boadilla 和Madrid。
    \item[带顶栏底栏] AnnArbor、Berlin、CambridgeUS、Copenhagen、Dresden、Ilmenau、Luebeck、Malmoe、Szeged 和Warsaw。
    \item[带侧栏] Berkeley、Goettingen、Hannover、Marburg 和 PaloAlto。
\end{description}

\begin{center}
    \footnotesize{\url{http://deic.uab.es/~iblanes/beamer_gallery/}}
    \footnotesize{\url{http://www.hartwork.org/beamer-theme-matrix/}}
\end{center}
\end{frame}

\begin{frame}\frametitle{主题DIY}
Beamer 的各部分的内容都可以自己定制和修改,和主题的划分
类似,可以从如下这三个方面来定制自己的主题:

\begin{description}
    \item[定制模板] 用\textbackslash setbeamertemplate 命令
    \item[定制颜色] 用\textbackslash setbeamercolor 命令
    \item[定制字体] 用\textbackslash setbeamerfont 命令
\end{description}
\end{frame}
