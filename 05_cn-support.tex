%************************************************
\section{中文支持}
\begin{frame}\frametitle{中文支持}
    \begin{itemize}
        \item<1-> CJK
            \begin{itemize}
              \item<2-> 一个德国人开发的中、日、韩文字处理包
              \item<3-> 是 \LaTeX{} 系统的一个宏包,比较通用
            \end{itemize}
        \item<4-> CCT
            \begin{itemize}
              \item<5-> 中科院张林波教授开发的中文系统
              \item<6-> 中文字体比较多,排版方式考虑中文使用习惯
              \item<7-> 缺点是需要进行预处理,引入其它LaTeX资源时会有一些问题
            \end{itemize}
        \item<8-> 天元
            \begin{itemize}
              \item<9-> 华东师大的肖刚、陈志杰等开发的中文 \TeX{}系统
            \end{itemize}
        \item<10-> \XeLaTeX{}
            \begin{itemize}
              \item<11-> 从底层支持中文
            \end{itemize}
    \end{itemize}
\end{frame}

\begin{frame}[fragile]\frametitle{CJK中文支持}
\begin{block}{CJK格式}
    \begin{verbatim}
\documentclass{article}
\usepackage{CJK}
\begin{document}
\begin{CJK*}{GBK}{kai}
这是中文楷体字。
\end{CJK*}
\end{document}
    \end{verbatim}
\end{block}
\end{frame}

\begin{frame}[fragile]\frametitle{CCT中文支持}
\begin{block}{老版本CCT格式}
    \begin{verbatim}
\documentclass{cctart}
\begin{document}
\kaishu 这是中文楷体字。
\end{document}
    \end{verbatim}
\end{block}

\begin{block}{新版本CCT格式}
    \begin{verbatim}
\documentclass[CJK]{cctart}
\begin{document}
\kaishu 这是中文楷体字。
\end{document}
    \end{verbatim}
\end{block}
\end{frame}

\begin{frame}[fragile]\frametitle{\XeLaTeX{} 中文支持}
\begin{block}{\XeLaTeX{} 格式1 (编码保存为UTF-8)}
    \begin{verbatim}
\documentclass{ctexart}
\begin{document}
中文宏包测试
\end{document}
    \end{verbatim}
\end{block}

\begin{block}{\XeLaTeX{}格式2}
    \begin{verbatim}
\documentclass{article}
\usepackage{ctex}
\begin{document}
中文宏包测试
\end{document}
    \end{verbatim}
\end{block}
\end{frame}
