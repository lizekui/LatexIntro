%************************************************
\section{数学公式}
\begin{frame}[fragile]\frametitle{传说中的方便插入公式?}
article中的两种公式:
    \begin{itemize}
      \hidark<1> \item 行内公式(inline mode)
        \begin{itemize}
          \hidark<2> \item \textbackslash ( \ldots \textbackslash )
		  \hidark<3> \item \textbackslash begin\{math\} \ldots \textbackslash end\{math\}
		  \hidark<4> \item \$ \ldots \$
        \end{itemize}
      \hidark<5> \item 行间公式(display mode)
        \begin{itemize}
          \hidark<6> \item \textbackslash begin\{equation\} \ldots \textbackslash end\{equation\}
		  \hidark<7> \item \textbackslash [ \ldots \textbackslash ]
		  \hidark<8> \item \textbackslash begin\{displaymath\} \ldots \textbackslash end\{displaymath\}
		  \hidark<9> \item \$\$ \ldots \$\$
        \end{itemize}
      \hidark<10> \item 多行公式
    \end{itemize}
\end{frame}

\begin{frame}[fragile]\frametitle{用例子说明两种公式的区别}
    \begin{block}{Input}
    \begin{verbatim}
I know that you know $1+1=2$, but I know $2-1=1$, which you
don't know. Now look at it $$2-1=1$$ I DO know more than you
    \end{verbatim}
    \end{block}
    \begin{block}{Output}
        I know that you know $1+1=2$, but I know $2-1=1$, which you don't know. Now look at it $$2-1=1$$ I DO know more than you.
    \end{block}
\end{frame}

\begin{frame}[fragile]\frametitle{分式、上下标和开方}
    \begin{block}{Input}
    \begin{verbatim}
$$\frac{2011}{2012}, x_1,x_2,\ldots,x_n, a^2+b^2=c^2,
x_1^2+x_2^2+\ldots+x_n^2=r^{100}, \sqrt{x+1},
\sqrt[3]{x^2+1}$$
    \end{verbatim}
    \end{block}
    \begin{block}{Output}
        $$\frac{2011}{2012}, x_1,x_2,\ldots,x_n, a^2+b^2=c^2, x_1^2+x_2^2+\ldots+x_n^2=r^{100}, \sqrt{x+1}, \sqrt[3]{x^2+1}$$
    \end{block}
\end{frame}

\begin{frame}[fragile]\frametitle{三角函数}
    \begin{block}{Input}
    \begin{verbatim}
$$\sin x, \cos x, \tan x, \arctan x, \sinh x, \cosh x,
\max x, \min x, \ln x, \log x, \log_2 x.$$
    \end{verbatim}
    \end{block}
    \begin{block}{Output}
        $$\sin x, \cos x, \tan x, \arctan x, \sinh x, \cosh x, \max x, \min x
        , \ln x, \log x, \log_2 x.$$
    \end{block}
\end{frame}

\begin{frame}[fragile]\frametitle{求和、极限和积分}
    \begin{block}{Input}
    \begin{verbatim}
$$\lim_{n \to \infty} a_n = 1, \sum_{n=1}^{\infty} n = 5050,
\int_{a}^{b}f(x) \mathrm{d}x = I$$
    \end{verbatim}
    \end{block}
    \begin{block}{Output}
        $$\lim_{n \to \infty} a_n = 1, \sum_{n=1}^{\infty} n = 5050,
        \int_{a}^{b}f(x) \mathrm{d}x = I$$
    \end{block}
\end{frame}

\begin{frame}[fragile]\frametitle{关系符号、希腊字母和部分数学标记}
    \begin{block}{Input}
    \begin{verbatim}
$$a \times b, c \div d, a<b , b=c, c \neq d, d > e,
e \geq f, f \leq g $$
$$\alpha\beta\gamma\delta\epsilon\varepsilon\xi\pi
\rho\sigma\eta\theta\phi\varphi\omega$$
$$|A|, \|A\|, \vec{a}, \overrightarrow{AB}, \tilde{x},
\widetilde{xyz}, \mathrm{sin}$$
    \end{verbatim}
    \end{block}
    \begin{block}{Output}
        $$a \times b, c \div d, a<b , b=c, c \neq d, d > e, e \geq f, f \leq g $$
        $$\alpha\beta\gamma\delta\epsilon\varepsilon\xi\pi\rho\sigma\eta
        \theta\phi\varphi\omega ,|A|, \|A\|, \vec{a}, \overrightarrow{AB}, \tilde{x},
        \widetilde{xyz}, \mathrm{sin}$$
    \end{block}
\end{frame}

\begin{frame}[fragile]\frametitle{矩阵}
  \begin{columns}
    \begin{column}{0.05\textwidth}
    \end{column}
    \begin{column}{0.45\textwidth}
    \begin{block}{Input}
    \begin{verbatim}
\begin{equation}
\left(
\begin{array}{ccc}
a_{11} & a_{12} & a_{13} \\
a_{21} & a_{22} & a_{23} \\
a_{31} & a_{32} & a_{33}
\end{array}
\right)
\end{equation}
    \end{verbatim}
    \end{block}
    \end{column}
    \begin{column}{0.025\textwidth}
    \end{column}
    \begin{column}{0.45\textwidth}
    \begin{block}{Output}
        \begin{equation}
        \left(
        \begin{array}{ccc}
        a_{11} & a_{12} & a_{13} \\
        a_{21} & a_{22} & a_{23} \\
        a_{31} & a_{32} & a_{33}
        \end{array}
        \right)
        \end{equation}
    \end{block}
    \end{column}
    \begin{column}{0.025\textwidth}
    \end{column}
  \end{columns}
\end{frame}

\begin{frame}[fragile]\frametitle{矩阵v2.0}
  \begin{columns}
    \begin{column}{0.05\textwidth}
    \end{column}
    \begin{column}{0.45\textwidth}
    \begin{block}{Input}
    \begin{verbatim}
\begin{equation}
\left\{
\begin{array}{c||c|c}
a_{11} & a_{12} & \\
\hline
a_{21} & & a_{23} \\
& a_{32} & a_{33}
\end{array}
\right)
\end{equation}
    \end{verbatim}
    \end{block}
    \end{column}
    \begin{column}{0.025\textwidth}
    \end{column}
    \begin{column}{0.45\textwidth}
    \begin{block}{Output}
        \begin{equation}
        \left\{
        \begin{array}{c||c|c}
        a_{11} & a_{12} & \\
        \hline
        a_{21} & & a_{23} \\
        & a_{32} & a_{33}
        \end{array}
        \right)
        \end{equation}
    \end{block}
    \end{column}
    \begin{column}{0.025\textwidth}
    \end{column}
  \end{columns}
\end{frame}

\begin{frame}[fragile]\frametitle{分段函数}
  \begin{columns}
    \begin{column}{0.05\textwidth}
    \end{column}
    \begin{column}{0.45\textwidth}
    \begin{block}{Input}
    \begin{verbatim}
\begin{equation}
\chi_A(x)=
\left\{
\begin{array}{ll}
1, & x \in A \\
0, & x \not\in A
\end{array}
\right.
\end{equation}
    \end{verbatim}
    \end{block}
    \end{column}
    \begin{column}{0.025\textwidth}
    \end{column}
    \begin{column}{0.45\textwidth}
    \begin{block}{Output}
        \begin{equation}
        \chi_A(x)=
        \left\{
        \begin{array}{ll}
        1, & x \in A \\
        0, & x \not\in A
        \end{array}
        \right.
        \end{equation}
    \end{block}
    \end{column}
    \begin{column}{0.025\textwidth}
    \end{column}
  \end{columns}
\end{frame}